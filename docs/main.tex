\documentclass{mimosis}

\usepackage{mathpartir}
\usepackage{amsthm}

\newtheorem{theorem}{Theorem}
\newtheorem{lemma}[theorem]{Lemma}

\title{Generlizing Bi-directional Typing}
\author{Xu Xue}

\begin{document}
\maketitle	
%\tableofcontents

\chapter{STLC with subtyping}

\section{Declarative System}

\begin{align*}
&\text{Types} \quad\quad &A, B ::=&~ \mathtt{Int} \mid \mathtt{Top} \mid A \rightarrow B\\
&\text{Expressions} \quad \quad &e::=&~ x \mid \lambda x . ~e \mid e_1~e_2 \mid e : A\\
&\text{Contexts} \quad\quad &\Gamma::=&~ . \mid \Gamma, x : A
\end{align*}

The second application rule is declarative \footnote{it differs from the original interpretation of bi-directional typing that targets for algorithmic system} since we need to guess the type $B$, with only partial information $A$. However, we can prove that it is implementable.

We need to justify two cases: $(\lambda f. f~1 : (\mathtt{Int} \rightarrow \mathtt{Int}))~(\lambda x.~x)$ and $(\lambda x.~x)~1$.

\begin{mathpar}
\inferrule*[lab=T-App1]
{\Gamma \vdash e_1 \Rightarrow A \rightarrow B \\
 \Gamma \vdash e_2 \Leftarrow A}
{\Gamma \vdash e_1~e_2 \Rightarrow B}


\inferrule*[lab=T-App2]
{\Gamma \vdash e_1 \Leftarrow A \rightarrow B \\
 \Gamma \vdash e_2 \Rightarrow A}
{\Gamma \vdash e_1~e_2 \Rightarrow B}
\end{mathpar}

\section{Algorithmic System}

\subsection{Syntax}

\begin{align*}
&\text{Types} \quad\quad &A, B ::=&~ \mathtt{Int} \mid \mathtt{Top} \mid A \rightarrow B \mid \boxed{e}\\
&\text{Expressions} \quad \quad &e::=&~ x \mid \lambda x . ~e \mid e_1~e_2 \mid e : A\\
&\text{Contexts} \quad\quad &\Gamma::=&~ . \mid \Gamma, x : A
\end{align*}

\subsection{Typing}

We introduce the typing judgment $\Gamma \vdash A \Rightarrow e \Rightarrow B$. It accepts three inputs: a context $\Gamma$, a checking type $A$ and an expression $B$ and returns an inferred type $B$.

\begin{mathpar}
\inferrule*[lab=T-Lit]
{\Gamma \vdash \mathtt{Int} <: A}
{\Gamma \vdash A \Rightarrow i \Rightarrow \mathsf{Int}}

\inferrule*[lab=T-Var]
{x : B \in \Gamma \\
 \Gamma \vdash B <: A}
{\Gamma \vdash A \Rightarrow x \Rightarrow B}

\inferrule*[lab=T-App]
{\Gamma \vdash \boxed{e_2} \rightarrow A \Rightarrow e_1 \Rightarrow C \rightarrow D}
{\Gamma \vdash A \Rightarrow e_1~e_2 \Rightarrow D}

\inferrule*[lab=T-Ann]
{\Gamma \vdash B \Rightarrow e \Rightarrow B \\
 \Gamma \vdash B <: A}
{\Gamma \vdash A \Rightarrow e : B \Rightarrow B}

\inferrule*[lab=T-Lam1]
{\Gamma \vdash \mathtt{Top} \Rightarrow e_1 \Rightarrow A \\
 \Gamma, x : A \vdash B \Rightarrow e \Rightarrow B'}
{\Gamma \vdash \boxed{e_1} \rightarrow B \Rightarrow \lambda x.~e \Rightarrow A \rightarrow B'}

\inferrule*[lab=T-Lam2]
{\Gamma, x : A \vdash B \Rightarrow e \Rightarrow B'}
{\Gamma \vdash A \rightarrow B \Rightarrow \lambda x.~e \Rightarrow A \rightarrow B'}

\inferrule*[lab=T-Lam3]
{ }
{\Gamma \vdash \mathtt{Top} \Rightarrow \lambda x.~e \Rightarrow A \rightarrow B}
\end{mathpar}

\subsection{Subtyping}

Typing and subtyping are mutually dependent. There's an observation that $\boxed{e}$ will only appear on the left of the subtyping.

\begin{mathpar}
\inferrule*[lab=S-Refl]	
{ }
{\Gamma \vdash \mathtt{Int} <: \mathtt{Int}}

\inferrule*[lab=S-Top]
{ }
{\Gamma \vdash A <: \mathtt{Top}}

\inferrule*[lab=S-Arr]
{\Gamma \vdash C <: A \\
 \Gamma \vdash B <: D}
{\Gamma \vdash A \rightarrow B <: C \rightarrow D}

\inferrule*[lab=S-Tele]
{\Gamma \vdash A \Rightarrow e \Rightarrow A'}
{\Gamma \vdash \boxed{e} <: A}

\end{mathpar}


\section{Metatheory}

\begin{lemma}[Generlizing infer mode]
$\Gamma \vdash e \Rightarrow A \cong \Gamma \vdash \mathtt{Top} \Rightarrow e \Rightarrow A$
\end{lemma}

\begin{lemma}[Generlizing check mode]
$\Gamma \vdash e \Leftarrow A \cong \Gamma \vdash A \Rightarrow e \Rightarrow A'$
\end{lemma}

\section{Problems Identified}

$\Gamma \vdash \mathtt{Top} \Rightarrow e \Rightarrow A'$ breaks the Lemma 2, since checked by \texttt{Top} tells us nothing and we cannot ensure it must infer a type.

\chapter{Related work}

Dunfield's bidirectional typing mentioned concluded this is a simultaneous input and output.


\end{document}