\def\OPTIONConf{0}%
\def\OPTIONArxiv{0}%
%
\documentclass{article}
\usepackage{fullpage}

\usepackage{mathpartir}

\usepackage{amsthm}

\newtheorem{theorem}{Theorem}
\newtheorem{lemma}[theorem]{Lemma}

\usepackage{amsmath}
\usepackage{marvosym}  % for \Pointinghand / \Hand in llproof

\usepackage{cite}

\usepackage{srcltx}
\usepackage{goodcharter}
\usepackage{euler}

\usepackage{jdunfield}
\usepackage{llproof}
\usepackage{rulelinks}
% strike through
\usepackage{soul}

\usepackage{hyperref}


\newcommand{\macroname}[1]{\texttt{\Backslash}\textvtt{#1}}

\newcommand{\unitty}{\tyname{1}}
\newcommand{\rulename}[1]{\textsf{#1}}
\newcommand{\elabtyperulename}[1]{\rulename{\textcolor{dDigPurple}{E#1}}}

\newrulecommand{EVar}{\elabtyperulename{var}}


\title{Contextual Type Inference}

\author{Xu Xue}


\begin{document}
\maketitle

\section{Declarative System}

\begin{align*}
&\text{Types} \quad\quad &A, B ::=&~ \mathtt{Int} \mid \mathtt{Top} \mid A \rightarrow B\\
&\text{Expressions} \quad \quad &e::=&~ x \mid \lambda x . ~e \mid e_1~e_2 \mid e : A\\
&\text{Contexts} \quad\quad &\Gamma::=&~ . \mid \Gamma, x : A
\end{align*}

The second application rule is declarative \footnote{it differs from the original interpretation of bi-directional typing that targets for algorithmic system} and not mode-correct since we need to guess the type $B$, with only partial information $A$. However, we can prove that it is implementable. We need to justify two cases: $(\lambda f. f~1 : (\mathtt{Int} \rightarrow \mathtt{Int}))~(\lambda x.~x)$ and $(\lambda x.~x)~1$.

\begin{mathpar}
\inferrule*[lab=T-App1]
{\Gamma \vdash e_1 \Rightarrow A \rightarrow B \\
 \Gamma \vdash e_2 \Leftarrow A}
{\Gamma \vdash e_1~e_2 \Rightarrow B}


\inferrule*[lab=T-App2]
{\Gamma \vdash e_1 \Leftarrow A \rightarrow B \\
 \Gamma \vdash e_2 \Rightarrow A}
{\Gamma \vdash e_1~e_2 \Rightarrow B}
\end{mathpar}

\section{Algorithmic System}

\subsection{Syntax}

\begin{align*}
&\text{Types with Holes}\quad\quad &A*, B* ::=&~ \mathtt{Int} \mid \mathtt{Top} \mid A* \rightarrow B* \mid \boxed{e}\\
&\text{Types} \quad\quad &A, B ::=&~ \mathtt{Int} \mid \mathtt{Top} \mid A \rightarrow B\\
&\text{Expressions} \quad \quad &e::=&~ x \mid \lambda x . ~e \mid e_1~e_2 \mid e : A\\
&\text{Contexts} \quad\quad &\Gamma::=&~ . \mid \Gamma, x : A
\end{align*}

\subsection{Typing}

We introduce the typing judgment $\Gamma \vdash A* \Rightarrow e \Rightarrow B$. It accepts three inputs: a context $\Gamma$, a checking type $A$ and an expression $B$ and returns an inferred type $B$.

There's a proposal to unify two lambda rules: simply abandon T-Lam1 and use T-Lam2 as the main rule, thus for the second example $(\lambda x.~x) ~ 1$, there'll be a case
\begin{mathpar}
\inferrule*[Right=T-Lam2]
{x : \boxed{1} \vdash Top \Rightarrow x \Rightarrow Int}
{. \vdash \boxed{1} \rightarrow Top \Rightarrow \lambda x . ~x \Rightarrow Int \rightarrow Int}
\end{mathpar}

\noindent Then, we need to add a constraint in the variable rule, which must infer a type in a canonical form (here, types without \boxed{e} is in canonical form).


\begin{mathpar}
\inferrule*[lab=T-Lit]
{\Gamma \vdash \mathtt{Int} <: A*}
{\Gamma \vdash A* \Rightarrow i \Rightarrow \mathsf{Int}}

\inferrule*[lab=T-Var]
{x : B \in \Gamma \\
 \Gamma \vdash B <: A*}
{\Gamma \vdash A* \Rightarrow x \Rightarrow B}

\inferrule*[lab=T-App]
{\Gamma \vdash \boxed{e_2} \rightarrow A* \Rightarrow e_1 \Rightarrow C \rightarrow D}
{\Gamma \vdash A* \Rightarrow e_1~e_2 \Rightarrow D}

\inferrule*[lab=T-Ann]
{\Gamma \vdash B \Rightarrow e \Rightarrow B \\
 \Gamma \vdash B <: A*}
{\Gamma \vdash A* \Rightarrow e : B \Rightarrow B}

\inferrule*[lab=T-Lam1]
{\Gamma \vdash \mathtt{Top} \Rightarrow e_1 \Rightarrow A \\
 \Gamma, x : A \vdash B* \Rightarrow e \Rightarrow B}
{\Gamma \vdash \boxed{e_1} \rightarrow B* \Rightarrow \lambda x.~e \Rightarrow A \rightarrow B}

\inferrule*[lab=T-Lam2]
{\Gamma, x : A \vdash B* \Rightarrow e \Rightarrow C}
{\Gamma \vdash A \rightarrow B* \Rightarrow \lambda x.~e \Rightarrow A \rightarrow C}
\end{mathpar}

\subsection{Subtyping}

Subtyping has the form $\Gamma \vdash A* <: B*$.
Typing and subtyping are mutually dependent. There's an observation that $\boxed{e}$ will only appear on the left of the subtyping.

\begin{mathpar}
\inferrule*[lab=S-Refl]	
{ }
{\Gamma \vdash \mathtt{Int} <: \mathtt{Int}}

\inferrule*[lab=S-Top]
{ }
{\Gamma \vdash A* <: \mathtt{Top}}

\inferrule*[lab=S-Arr]
{\Gamma \vdash C* <: A* \\
 \Gamma \vdash B* <: D*}
{\Gamma \vdash A* \rightarrow B* <: C* \rightarrow D*}

\inferrule*[lab=S-Tele]
{\Gamma \vdash A* \Rightarrow e \Rightarrow A}
{\Gamma \vdash \boxed{e} <: A*}

\end{mathpar}

\section{Examples}

\subsection{$(\lambda f . ~ f ~ 1) : ((Int \rightarrow Int) \rightarrow Int)~ (\lambda x. ~x)$}

This case only uses the T-Lam2 case, which is similar to the check mode for the $\lambda.x ~e$. ($\lambda x.~x$ cannot infer a type thus it skips the T-Lam1).

\begin{mathpar}
\small
\inferrule*[Right=T-App]
%%%%%% T-App Premise
{
\inferrule*[Right=T-Ann]
{
\inferrule*[Right=T-Lam2]{ }{. \vdash (Int \rightarrow Int) \rightarrow Int \Rightarrow \lambda f . ~ f ~ 1 \Rightarrow (Int \rightarrow Int) \rightarrow Int} \quad \quad \quad \quad
\inferrule*[Right=S-Arr]{ }{. \vdash (Int \rightarrow Int) \rightarrow Int <: \boxed{\lambda x. ~x} \rightarrow Top}
}
{. \vdash \boxed{\lambda x. ~x} \rightarrow Top \Rightarrow (\lambda f . ~ f ~ 1) : ((Int \rightarrow Int) \rightarrow Int) \Rightarrow (Int \rightarrow Int) \rightarrow Int}
}
%%%%%% T-App Conclusion
{. \vdash Top \Rightarrow (\lambda f . ~ f ~ 1) : ((Int \rightarrow Int) \rightarrow Int)~ (\lambda x. ~x) \Rightarrow Int}
\end{mathpar}

\noindent For the second premise, we have

\begin{mathpar}
\inferrule*[Right=S-Arr]
{
\inferrule*[Right=S-Tele]
{. \vdash Int \rightarrow Int \Rightarrow \lambda x. ~x \Rightarrow Int \rightarrow Int}
{ . \vdash \boxed{\lambda x. ~x} <: Int \rightarrow Int} \\
. \vdash Int <: Top
}
{. \vdash (Int \rightarrow Int) \rightarrow Int <: \boxed{\lambda x. ~x} \rightarrow Top}
\end{mathpar}


\subsection{$(\lambda x. ~x)~1$}

This case is used in ``let arguments go first", where the type of arguments can be known.

\begin{mathpar}
\inferrule*[Right=T-App]
{
\inferrule*[Right=T-Lam1]
{. \vdash Top \Rightarrow 1 \Rightarrow Int \\ x : Int \vdash Top \Rightarrow x \Rightarrow Int}
{. \vdash \boxed{1} \rightarrow Top \Rightarrow \lambda x. ~x \Rightarrow Int \rightarrow Int}
}
{. \vdash Top \Rightarrow (\lambda x. ~x)~1 \Rightarrow Int}
\end{mathpar}




\section{Metatheory}

\begin{lemma}[Subtyping]
If $\Gamma \vdash A \Rightarrow e \Rightarrow B$, then $\Gamma \vdash B <: A$.
\end{lemma}

\begin{lemma}[Generlizing infer mode]
$\Gamma \vdash e \Rightarrow A \cong \Gamma \vdash \mathtt{Top} \Rightarrow e \Rightarrow A$
\end{lemma}

This maybe wrong if the declarative inference is not unique.

\begin{lemma}[Generlizing check mode]
$\Gamma \vdash e \Leftarrow A \cong \Gamma \vdash A \Rightarrow e \Rightarrow A$
\end{lemma}

\begin{theorem}[Completeness]
If $\Gamma \vdash e \Rightarrow A$, then $\Gamma \vdash Top \Rightarrow e \Rightarrow B$ and $B <: A$.
\end{theorem}

\begin{theorem}[Soundness]

\end{theorem}



\section{Problems Identified}

\begin{enumerate}
	\item $\Gamma \vdash \mathtt{Top} \Rightarrow e \Rightarrow A'$ breaks the Lemma 2, since checked by \texttt{Top} tells us nothing and we cannot ensure it must infer a type.
	\item \st{The metatheory only tells that it subsumes the traditional bidirectional typing, but it hasn't shown that we can infer more types}.
	\item We need to find a way to normalize/categorize the type, otherwise, we will get the final inferred type which is \boxed{e}.
	\item There is a concern about rule T-Lam3, whose type has no meaning.
\end{enumerate}

\section{Discussions}

\subsection{Top Type}
In traditional system bidirectional typing, we have the rule and/or property that $\Gamma \vdash e \Leftarrow Top$, right?

\paragraph{Remark} We should start with whether the declarative system supports such a property. For now, it is not supported by the rules in declarative one and may lose the property of check subsumption (the ideal one). 

\subsection{Let-binding and lambda binding}

Appending one rule like this simply solves the $(\lambda x .~x) 1$ where arguments appear as a binding in the function body.

\begin{mathpar}
\inferrule*[lab=T-Let]
{\Gamma \vdash e \Rightarrow A \\ \Gamma, x : A \vdash e' \Rightarrow B}
{\Gamma \vdash let ~x = e ~in~ e' \Rightarrow B}

\inferrule*[lab=T-App-Dumb]
{\Gamma \vdash e_2 \Rightarrow A \\ \Gamma, x : A \vdash e_2 \Rightarrow B}
{\Gamma \vdash (\lambda x . ~e_1) ~e_2 \Rightarrow B}
\end{mathpar}

\paragraph{Remark} The typing rule for let can easily deal with nested lets, but not with the counter rule in the right. This desugar can work relatively well with other systems, but not in a bidirectional typing style.

\subsection{Type flows out}

There's a variant of bidirectional typing that can type check the term $\lambda x. ~x + x$, where the inner type flows out, which seems not possible in our system.

\subsection{Uniqueness of inference, check and another possiblity}

Snow argues the rule T-App2 is not correct, in terms of the merge operator.

\begin{mathpar}
\inferrule*[lab=T-App2]
{\Gamma \vdash e_1 \Leftarrow A \rightarrow B \\
 \Gamma \vdash e_2 \Rightarrow A}
{\Gamma \vdash e_1~e_2 \Rightarrow B}
\end{mathpar}

\subsection{Another Style of Typing}

\begin{align*}
&\text{Types} \quad\quad &A, B ::=&~ \mathtt{Int} \mid \mathtt{Top} \mid A \rightarrow B \mid \boxed{e}\\
&\text{Types Normal Form} \quad\quad &A*, B* ::=&~ \mathtt{Int} \mid \mathtt{Top} \mid A* \rightarrow B* \\
&\text{Expressions} \quad \quad &e::=&~ x \mid \lambda x . ~e \mid e_1~e_2 \mid e : A\\
&\text{Contexts} \quad\quad &\Gamma::=&~ . \mid \Gamma, x : A
\end{align*}

Note that T-Var1 and T-Var2 are not complete in syntax and may not cover all cases.
\paragraph{Remark} It seems to complicate the story since we put the \boxed{e} in the context instead of propogating it through the checking place.

\begin{mathpar}
\inferrule*[lab=T-Lit]
{\Gamma \vdash \mathtt{Int} <: A}
{\Gamma \vdash A \Rightarrow i \Rightarrow \mathsf{Int}}

\inferrule*[lab=T-Var1]
{x : B* \in \Gamma \\
 \Gamma \vdash B* <: A}
{\Gamma \vdash A \Rightarrow x \Rightarrow B*}

\inferrule*[lab=T-Var2]
{x : \boxed{e} \in \Gamma \\ \Gamma \vdash Top \Rightarrow e \Rightarrow B* \\
 \Gamma \vdash B* <: A}
{\Gamma \vdash A \Rightarrow x \Rightarrow B*}

\inferrule*[lab=T-App]
{\Gamma \vdash \boxed{e_2} \rightarrow A \Rightarrow e_1 \Rightarrow C \rightarrow D}
{\Gamma \vdash A \Rightarrow e_1~e_2 \Rightarrow D}

\inferrule*[lab=T-Ann]
{\Gamma \vdash B \Rightarrow e \Rightarrow B \\
 \Gamma \vdash B <: A}
{\Gamma \vdash A \Rightarrow e : B \Rightarrow B}

\inferrule*[lab=T-Lam]
{\Gamma, x : A \vdash B \Rightarrow e \Rightarrow B'}
{\Gamma \vdash A \rightarrow B \Rightarrow \lambda x.~e \Rightarrow A \rightarrow B'}
\end{mathpar}

\subsection{Minimal type property}
We expect a property from declaretive system to algorithmic system, it that the algorithmic one always computes a minimal type.

\begin{lemma}[Minimal type]
If $\Gamma \vdash e \Rightarrow A$, then $\Gamma \vdash Top \Rightarrow e \Rightarrow B$ and $B <: A$.	
\end{lemma}

\paragraph{Remark} this property/lemma is actually the completeness lemma of two systems.




\section{Related work}

\subsection{Bidirectional Typing \cite{dunfield2021bidirectional}}
It mentioned concluded this is a simultaneous input and output. 

\subsection{Round-trip Type Checking \cite{polikarpova2016program}}

\begin{center}
	\url{https://people.csail.mit.edu/asolar/SynthesisCourse/Lecture16.htm}	
\end{center}

\begin{quote}
Bidirectional type
propagation is “all-or-nothing”: once a checking problem for
a term cannot be decomposed perfectly into checking problems
for its subterms, the system abandons all information about the
goal type and switches to purely bottom-up inference. Our insight is that some information from the goal type can be retained in the bottom-up phase, leading to more local error detection.
\end{quote}

\subsection{Let arguments go first \cite{xie2018let}}

\begin{mathpar}
\inferrule*[lab=T-App]
{\Gamma \vdash e_2 \Rightarrow A \\ \Gamma \mid \Psi, A \vdash e_1 \Rightarrow A \rightarrow B}
{\Gamma \mid \Psi \vdash e_1 ~ e_2 \Rightarrow B}

\inferrule*[lab=T-Lam]
{\Gamma, x : A \mid \Psi \vdash e \Rightarrow B}
{\Gamma \mid \Psi, A \vdash \lambda x~. e \Rightarrow A \rightarrow B}
\end{mathpar}



\bibliography{ref}
\bibliographystyle{plain}

\end{document}


% Local Variables: 
% mode: latex
% TeX-master: example
% End: 
