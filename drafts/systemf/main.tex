\documentclass{article}
\usepackage{fullwidth}
\usepackage{graphicx} % Required for inserting images

\usepackage{mathpartir}
\usepackage{amsmath}
\usepackage{amssymb}
\usepackage{mathtools}
\usepackage{xfrac,unicode-math}
% \usepackage{amsthm}
\usepackage{amsthm}
\newtheorem{lemma}{Lemma}

\title{Contextual Type Inference (System F)}
\author{Xu Xue}

\begin{document}

\maketitle

\section{Syntax}

\subsection{Decl.}

\begin{align*}
&\text{Types} \quad\quad &A, B, C, D, T ::=&~ a \mid \mathtt{Int} \mid A \rightarrow B \mid \forall a.~A\\
&\text{Expressions} \quad \quad &e::=&~ x \mid \lambda x . ~e \mid e_1~e_2 \mid e : A \mid \Lambda a.~e \mid e[A]\\
&\text{Contexts} \quad\quad &\Gamma::=&~ . \mid \Gamma, x : A \mid \Gamma, a\\
&\text{Counters} \quad\quad &j ::=&~ n \mid \infty\\
\end{align*}

\subsection{Algo.}

\begin{align*}
    &\text{Hints} \quad\quad &H ::=&~ A \mid \square \mid \boxed{e} \rightarrow H\\
    &\text{Partial Values} \quad \quad &p::=&~ i \mid x \mid e : A\\
    &\text{Contexts} \quad\quad &\Gamma,\Delta::=&~ . \mid \Gamma, x : A \mid \Gamma, a \mid \Gamma, \hat{a} \mid \Gamma, \hat{a}:=A\\
    &\text{Mapping}\quad\quad &\phi ::=&~. \mid \phi, \hat{a}:=A \mid \phi, a
\end{align*}

\subsection{Operations on Counters}

\begin{align*}
    (S~n)- &= n \\
    \infty - &= \infty \\
    n+ &= S~n \\
    \infty + &= \infty
\end{align*}

\section{Decl.}

\subsection{Typing}

\begin{mathpar}
    \inferrule*[lab=Int]
    { }
    {\Gamma \vdash_0 i : \mathtt{Int}}

    \inferrule*[lab=Var]
    {x : A \in \Gamma}
    {\Gamma \vdash_0 x : A}

    \inferrule*[lab=Ann]
    {\Gamma \vdash_{\infty} e : A}
    {\Gamma \vdash_0 (e : A) : A}

    \inferrule*[lab=Lam]
    {\Gamma, x : A \vdash_{j-} e : B}
    {\Gamma \vdash_{j} \lambda x.~ e : A \rightarrow B}

    \inferrule*[lab=App1]
    {\Gamma \vdash_0 e_1 : A \rightarrow B \\
    \Gamma \vdash_{\infty} e_2 : A}
    {\Gamma \vdash_n e_1~e_2 : B}

    \inferrule*[lab=App2]
    {\Gamma \vdash_{j+} e_1 : A \rightarrow B \\
    \Gamma \vdash_0 e_2 : A}
    {\Gamma \vdash_j e_1~e_2 : B}

    \inferrule*[lab=Sub]
    {\Gamma \vdash_0 e : B \\
    \Gamma \vdash B <:_j A}
    {\Gamma \vdash_j e : A}

    \inferrule*[lab=TAbs]
    {\Gamma, X \vdash_j e : A}
    {\Gamma \vdash_j \Lambda X.~e : \forall X.~A}

    \inferrule*[lab=TApp]
    {\Gamma \vdash_j e : \forall X.~B}
    {\Gamma \vdash_j e[A] : [A/X]B}
\end{mathpar}

\subsection{Subtyping}

\begin{mathpar}
    \inferrule*[lab=S-Refl]
    { }
    {\Gamma \vdash_0 A <: A}

    \inferrule*[lab=S-Int]
    { }
    {\Gamma \vdash_{\infty} \mathtt{Int} <: \mathtt{Int}}

    \inferrule*[lab=S-Arr]
    {\Gamma \vdash_{\infty} C <: A \\
     \Gamma \vdash_{j-} B <: D}
    {\Gamma \vdash_j A \rightarrow B <:C \rightarrow D}

    \inferrule*[lab=S-$\forall$]
    {\Gamma , a \vdash_{\infty} A <: B}
    {\Gamma \vdash_{\infty} \forall a.~A <: \forall a.~B}

    \inferrule*[lab=S-$\forall$L]
    {\Gamma \vdash B \\
     \Gamma \vdash_{S~n} [B/a] A <: C}
    {\Gamma \vdash_{S~ n} \forall a. ~A <: C}
\end{mathpar}

\section{Algo.}

\subsection{Typing: $\boxed{\Gamma \vdash H \Rightarrow e \Rightarrow A}$}

\begin{mathpar}
\inferrule*[lab=T-Lit]
{ }
{\Gamma \vdash \square \Rightarrow i \Rightarrow \mathsf{Int}}

\inferrule*[lab=T-Var]
{x : A \in \Gamma}
{\Gamma \vdash \square \Rightarrow x \Rightarrow A}

\inferrule*[lab=T-Ann]
{\Gamma \vdash A \Rightarrow e \Rightarrow B}
{\Gamma \vdash \square \Rightarrow e : A \Rightarrow A}

\inferrule*[lab=T-TAbs]
{\Gamma, a \vdash \square \Rightarrow e \Rightarrow A}
{\Gamma \vdash \square \Rightarrow \Lambda a.~e \Rightarrow \forall a.~A}

\inferrule*[lab=T-App]
{\Gamma \vdash \boxed{e_2} \rightarrow H \Rightarrow e_1 \Rightarrow A \rightarrow B}
{\Gamma \vdash H \Rightarrow e_1~e_2 \Rightarrow B}

\inferrule*[lab=T-Lam1]
{\Gamma, x : A \vdash B \Rightarrow e \Rightarrow C}
{\Gamma \vdash A \rightarrow B \Rightarrow \lambda x.~e \Rightarrow A \rightarrow C}

\inferrule*[lab=T-Lam2]
{\Gamma \vdash \square \Rightarrow e_2 \Rightarrow A\\
 \Gamma , x : A \vdash H \Rightarrow e \Rightarrow B}
{\Gamma \vdash \boxed{e_2} \rightarrow H \Rightarrow \lambda x.~e \Rightarrow A \rightarrow B}

\inferrule*[lab=T-Sub]
{\Gamma \vdash \square \Rightarrow p \Rightarrow A \\
 \Gamma \vdash A <: H \dashv \Gamma \rightsquigarrow B}
{\Gamma \vdash H \Rightarrow p \Rightarrow B}

\inferrule*[lab=T-TApp]
{\Gamma \vdash H \Rightarrow e \Rightarrow \forall X.~B}
{\Gamma \vdash H \Rightarrow e[A] \Rightarrow [A/X]B}

\end{mathpar}

\subsection{Subtyping: $\boxed{\Gamma \vdash A <: H \dashv \Delta \rightsquigarrow A}$}

\begin{mathpar}
\inferrule*[lab=S-Int]
{ }
{\Gamma \vdash \mathtt{Int} <: \mathtt{Int} \dashv \Gamma \rightsquigarrow \mathtt{Int}}

\inferrule*[lab=S-Ex-L]
{FV_{ex}(A) = . \\ \Gamma \vdash A}
{\Gamma \vdash \hat{a} <: A \dashv [A/\hat{a}]\Gamma \rightsquigarrow A}

\inferrule*[lab=S-Ex-R]
{FV_{ex}(A) = . \\ \Gamma \vdash A}
{\Gamma \vdash A <: \hat{a} \dashv [A/\hat{a}]\Gamma \rightsquigarrow A}

\inferrule*[lab=S-Arr]
{\Gamma \vdash C <: A \dashv \Delta_1 \rightsquigarrow A' \\
 \Delta_1 \vdash B <: D \dashv \Delta_2 \rightsquigarrow D'}
{\Gamma \vdash A \rightarrow B <: C \rightarrow D \dashv \Delta_2 \rightsquigarrow A' \rightarrow D'}

\inferrule*[lab=S-Hole-No-Ex]
{ FV_{ex}(A) = \emptyset \\
\Gamma \vdash A \Rightarrow e \Rightarrow C\\
 \Gamma \vdash B <: H \dashv \Delta \rightsquigarrow D}
{\Gamma \vdash A \rightarrow B <: \boxed{e} \rightarrow H \dashv \Delta \rightsquigarrow A \rightarrow D}

\inferrule*[lab=S-Hole-Ex]
{ FV_{ex}(A) != \emptyset \\
\Gamma \vdash \square \Rightarrow e \Rightarrow C\\
\Gamma \vdash C <: A \dashv \Delta_1 \rightsquigarrow A' \\
\Delta_1 \vdash B <: H \dashv \Delta_2 \rightsquigarrow D}
{\Gamma \vdash A \rightarrow B <: \boxed{e} \rightarrow H \dashv \Delta_2 \rightsquigarrow A' \rightarrow D}

\inferrule*[lab=S-All]
{\Gamma, a \vdash A <: B \dashv \Delta_1, a, \Delta_2 \rightsquigarrow C}
{\Gamma \vdash \forall a. A <: \forall a. B \dashv \Delta_1 \rightsquigarrow \forall a. C}

\inferrule*[lab=S-All-L]
{\Gamma, \hat{a} \vdash [\hat{a}/a]A <: \boxed{e} \rightarrow H \dashv \Delta_1, \hat{a}, \Delta_2 \rightsquigarrow B}
{\Gamma \vdash \forall a. A <: \boxed{e} \rightarrow H \dashv \Delta_1 \rightsquigarrow B}

\end{mathpar}
\end{document}