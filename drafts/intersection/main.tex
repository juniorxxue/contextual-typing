\documentclass{article}
\usepackage{fullwidth}
\usepackage{graphicx} % Required for inserting images

\usepackage{mathpartir}
\usepackage{amsmath}
\usepackage{amssymb}
\usepackage{mathtools}
\usepackage{xfrac,unicode-math}
% \usepackage{amsthm}
\usepackage{amsthm}
\newtheorem{lemma}{Lemma}

\title{Contextual Type Inference (Intersection Types)}
\author{Xu Xue}

\begin{document}

\maketitle

\section{Syntax}

\begin{align*}
&\text{Types} \quad\quad &A, B, C, D, T ::=&~ \mathtt{Int} \mid \mathtt{Top} \mid A \rightarrow B \mid A~\&~B\\
&\text{Expressions} \quad \quad &e::=&~ x \mid \lambda x . ~e \mid e_1~e_2 \mid e : A\\
&\text{Partial Values} \quad \quad &p::=&~ i \mid x \mid e : A\\
&\text{Contexts} \quad\quad &\Gamma::=&~ . \mid \Gamma, x : A\\
&\text{Check Counters} \quad\quad &j ::=&~ Z \mid \infty \mid S_{\Leftarrow}~j \\
&\text{Counters} \quad\quad &i ::= &~ \beta ~j \mid S_{\Rightarrow}~i\\
&\text{Hints} \quad\quad &H ::=&~ \square \mid A \mid \boxed{e} \rightarrow H\\
\end{align*}

\section{Decl.}

\subsection{Typing}

\begin{mathpar}
    \inferrule*[lab=Int]
    { }
    {\Gamma \vdash_{\beta ~Z} i : Int}

    \inferrule*[lab=Var]
    {x : A \in \Gamma}
    {\Gamma \vdash_{\beta ~Z} x : A}

    \inferrule*[lab=Ann]
    {\Gamma \vdash_{\beta ~\infty} e : A}
    {\Gamma \vdash_{\beta ~Z} (e : A) : A}

    \inferrule*[lab=Lam]
    {\Gamma, x : A \vdash_{i-} e : B}
    {\Gamma \vdash_{i} \lambda x.~ e : A \rightarrow B}

    \inferrule*[lab=App1]
    {\Gamma \vdash_{\beta ~S_{\Leftarrow}~j} e_1 : A \rightarrow B \\
    \Gamma \vdash_{\beta ~\infty} e_2 : A}
    {\Gamma \vdash_{\beta ~j} e_1~e_2 : B}

    \inferrule*[lab=App2]
    {\Gamma \vdash_{S_{\Rightarrow}~i} e_1 : A \rightarrow B \\
    \Gamma \vdash_{\beta ~Z} e_2 : A}
    {\Gamma \vdash_i e_1~e_2 : B}

    \inferrule*[lab=Sub]
    {\Gamma \vdash_{\beta ~Z} e : A \\
    A <:_{i} B}
    {\Gamma \vdash_i e : B}

    \inferrule*[lab=And]
    {\Gamma \vdash_{\beta ~\infty} e : A \\
     \Gamma \vdash_{\beta ~\infty} e : B}
    {\Gamma \vdash_{\beta ~\infty} e : A ~\&~ B}
\end{mathpar}

\subsection{Subtyping}

\begin{mathpar}
    \inferrule*[lab=S-Refl]
    { }
    {A <:_{\beta ~Z} A}

    \inferrule*[lab=S-Int]
    { }
    {Int <:_{\beta ~\infty} Int}

    \inferrule*[lab=S-Top]
    { }
    {A <:_{\beta ~\infty} \mathtt{Top}}

    \inferrule*[lab=S-Arr]
    {C <:_{\beta~\infty} A \\
     B <:_{i--} D}
    {A \rightarrow B <:_i C \rightarrow D}

    \inferrule*[lab=S-And-L]
    {A <:_i C}
    {A ~\&~ B <:_i C}

    \inferrule*[lab=S-And-R]
    {B <:_i C}
    {A ~\&~ B <:_i C}

    \inferrule*[lab=S-And]
    {A <:_{\beta~\infty} B \\
     A <:_{\beta~\infty} C}
    {A <:_{\beta~\infty} B ~\&~ C}
\end{mathpar}

\section{Algo.}

\subsection{Typing}

\begin{mathpar}
    \inferrule*[lab=T-Lit]
    { }
    {\Gamma \vdash \square \Rightarrow i \Rightarrow \mathsf{Int}}
    
    \inferrule*[lab=T-Var]
    {x : A \in \Gamma}
    {\Gamma \vdash \square \Rightarrow x \Rightarrow A}
    
    \inferrule*[lab=T-Ann]
    {\Gamma \vdash A \Rightarrow e \Rightarrow B}
    {\Gamma \vdash \square \Rightarrow e : A \Rightarrow A}
    
    \inferrule*[lab=T-App]
    {\Gamma \vdash \boxed{e_2} \mapsto H \Rightarrow e_1 \Rightarrow A \rightarrow B}
    {\Gamma \vdash H \Rightarrow e_1~e_2 \Rightarrow B}
    
    \inferrule*[lab=T-Lam1]
    {\Gamma, x : A \vdash B \Rightarrow e \Rightarrow C}
    {\Gamma \vdash A \rightarrow B \Rightarrow \lambda x.~e \Rightarrow A \rightarrow C}
    
    \inferrule*[lab=T-Lam2]
    {\Gamma \vdash \square \Rightarrow e_2 \Rightarrow A \\
     \Gamma , x : A \vdash H \Rightarrow e \Rightarrow B}
    {\Gamma \vdash \boxed{e_2} \rightarrow H \Rightarrow \lambda x.~e \Rightarrow A \rightarrow B}

\inferrule*[lab=T-Sub]
{\Gamma \vdash \square \Rightarrow p \Rightarrow A \\
 \Gamma \vdash A <: H \rightsquigarrow B}
{\Gamma \vdash H \Rightarrow p \Rightarrow B}

\inferrule*[lab=T-And]
{\Gamma \vdash A \Rightarrow e \Rightarrow A \\
 \Gamma \vdash B \Rightarrow e \Rightarrow B}
{\Gamma \vdash A ~\&~ B \Rightarrow e \Rightarrow A ~\&~ B}
\end{mathpar}

\subsection{Subtyping}

\begin{mathpar}
\inferrule*[lab=S-Int]
{ }
{\Gamma \vdash Int <: Int \rightsquigarrow Int}

\inferrule*[lab=S-Top]
{ }
{\Gamma \vdash A <: Top \rightsquigarrow Top}

\inferrule*[lab=S-Square]
{ }
{\Gamma \vdash A <: \square \rightsquigarrow A}

\inferrule*[lab=S-Arr]
{\Gamma \vdash C <: A \rightsquigarrow A \\
 \Gamma \vdash B <: D \rightsquigarrow D}
{\Gamma \vdash A \rightarrow B <: C \rightarrow D \rightsquigarrow C \rightarrow D}

\inferrule*[lab=S-Hole]
{\Gamma \vdash A \Rightarrow e \Rightarrow A \\
 \Gamma \vdash B <: H \rightsquigarrow D}
{\Gamma \vdash A \rightarrow B <: \boxed{e} \rightarrow H \rightsquigarrow A \rightarrow D}

\inferrule*[lab=S-And-L]
{\Gamma \vdash A <: H \rightsquigarrow C}
{\Gamma \vdash A ~\&~ B <: H \rightsquigarrow C}

\inferrule*[lab=S-And-R]
{\Gamma \vdash B <: H \rightsquigarrow C}
{\Gamma \vdash A ~\&~ B <: H \rightsquigarrow C}

\inferrule*[lab=S-And]
{\Gamma \vdash A <: B \rightsquigarrow B \\
 \Gamma \vdash A <: C \rightsquigarrow C}
{\Gamma \vdash A <: B ~\&~ C \rightsquigarrow B ~\&~ C}
\end{mathpar}   
\end{document}