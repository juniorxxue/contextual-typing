\section{Alternative Algo}

\begin{frame}{Syntax}
\begin{align*}
&\text{Hints}\quad\quad &H ::=&~ A \mid \boxed{e} \mapsto H\\
&\text{Normal Types} \quad\quad &A, B ::=&~ \mathtt{Int} \mid \mathtt{Top} \mid A \rightarrow B
\end{align*}

I'm trying to think could it have something like
$$
\boxed{e} \rightarrow A \rightarrow \boxed{e} \rightarrow ... or \quad A \rightarrow \boxed{e} \rightarrow ...
$$

I don't think so, since there're several ways to modify the hint

\begin{itemize}
	\item Lambda, only eliminate the head
	\item Ann, replace the whole hint with a normal type (annotation)
	\item App, chain/append a hole on top of a hint
\end{itemize}


\end{frame}

\begin{frame}{Typing: \boxed{\Gamma \vdash H \Rightarrow e \Rightarrow A}}
\begin{mathpar}
\small
\inferrule*[lab=T-Lit]
{\Gamma \vdash \mathtt{Int} <: H}
{\Gamma \vdash H \Rightarrow i \Rightarrow \mathsf{Int}}

\inferrule*[lab=T-Var]
{x : A \in \Gamma \\
 \Gamma \vdash A <: H}
{\Gamma \vdash H \Rightarrow x \Rightarrow A}

\inferrule*[lab=T-App]
{\Gamma \vdash \boxed{e_2} \mapsto H \Rightarrow e_1 \Rightarrow A \rightarrow B}
{\Gamma \vdash H \Rightarrow e_1~e_2 \Rightarrow B}

\inferrule*[lab=T-Ann]
{\Gamma \vdash A \Rightarrow e \Rightarrow A \\
 \Gamma \vdash A <: H}
{\Gamma \vdash H \Rightarrow e : A \Rightarrow A}

\inferrule*[lab=T-Lam1]
{\Gamma \vdash \mathtt{Top} \Rightarrow e_1 \Rightarrow A \\
 \Gamma, x : A \vdash H \Rightarrow e \Rightarrow B}
{\Gamma \vdash \boxed{e_1} \mapsto H \Rightarrow \lambda x.~e \Rightarrow A \rightarrow B}

\inferrule*[lab=T-Lam2]
{\Gamma, x : A \vdash B \Rightarrow e \Rightarrow C}
{\Gamma \vdash A \rightarrow B \Rightarrow \lambda x.~e \Rightarrow A \rightarrow C}
\end{mathpar}    
\end{frame}

\begin{frame}{Subtyping: \boxed{\Gamma \vdash A <: H}}
\begin{mathpar}
\inferrule*[lab=S-Refl]	
{ }
{\Gamma \vdash \mathtt{Int} <: \mathtt{Int}}

\inferrule*[lab=S-Top]
{ }
{\Gamma \vdash A <: \mathtt{Top}}

\inferrule*[lab=S-Arr]
{\Gamma \vdash C <: A \\
 \Gamma \vdash B <: D}
{\Gamma \vdash A \rightarrow B <: C \rightarrow D}

\inferrule*[lab=S-Chain]
{\Gamma \vdash A \Rightarrow e \Rightarrow A	 \\
 \Gamma \vdash B <: H}
{\Gamma \vdash A \rightarrow B <: \boxed{e} \mapsto H}
\end{mathpar}    
\end{frame}

\begin{frame}{Properties}
\begin{lemma}[$\vdash$ to $<:$]
If $\Gamma \vdash H \Rightarrow e \Rightarrow A$ then $\Gamma \vdash A <: H$.
\end{lemma}
\begin{proof}
Induction on $\Gamma \vdash H \Rightarrow e \Rightarrow A$; other cases are trivial, we only focus on first case of lambda
\end{proof}

\begin{itemize}
	\item \textbf{case lam1:} goal: $\Gamma \vdash A \rightarrow B <: \boxed{e_1} \mapsto H$, given $\Gamma \vdash Top \Rightarrow e_1 \Rightarrow A$ and $\Gamma, x : A \vdash H \Rightarrow e \Rightarrow B$;
	\item that is to prove $\Gamma \vdash A \Rightarrow e \Rightarrow A$ and $\Gamma \vdash B <: H$;
	\item with ind hypo, we can have $\Gamma, x : A \vdash B <: H$;
\end{itemize}

\noindent\makebox[\linewidth]{\rule{\paperwidth}{0.4pt}}

\begin{lemma}[No hint and full hint]
If $\Gamma \vdash Top \Rightarrow e \Rightarrow A$, then $\Gamma \vdash A \Rightarrow e \Rightarrow A$.
\end{lemma}

\end{frame}

\begin{frame}{Proof of No hint and full hint}
\begin{lemma}[No hint and full hint]
If $\Gamma \vdash Top \Rightarrow e \Rightarrow A$, then $\Gamma \vdash A \Rightarrow e \Rightarrow A$.
\end{lemma}
The tricky case is app:
\begin{itemize}
	\item goal: $\Gamma \vdash A \Rightarrow e_1~e_2 \Rightarrow A$, given $\Gamma \vdash Top \Rightarrow e_1~e_2 \Rightarrow A$;
	\item that is to prove $\Gamma \vdash \boxed{e_2} \mapsto A \Rightarrow e_1 \Rightarrow \Delta \rightarrow A$;
	\item and we have $\Gamma \vdash \boxed{e_2} \rightarrow Top \Rightarrow e_1 \Rightarrow \Pi \rightarrow A$;
\end{itemize}
\end{frame}

\begin{frame}{Proof of No hint and full hint (Cont.)}
\begin{lemma}[No hint and full hint (Gen)]
If $\Gamma \vdash H \Rightarrow e \Rightarrow A$, then $\Gamma \vdash A \Rightarrow e \Rightarrow A$.
\end{lemma}

We generalize it to the max; the app case will be
\begin{itemize}
	\item goal: $\Gamma \vdash A \Rightarrow e_1~e_2 \Rightarrow A$, given $\Gamma \vdash H \Rightarrow e_1~e_2 \Rightarrow A$;
	\item that is to prove $\Gamma \vdash \boxed{e_2} \mapsto A \Rightarrow e_1 \Rightarrow \Delta \rightarrow A$;
	\item and we have $\Gamma \vdash \boxed{e_2} \rightarrow H \Rightarrow e_1 \Rightarrow \Pi \rightarrow A$;
	\item by ind hypo, we have $\Gamma \vdash \Pi \rightarrow A \Rightarrow e_1 \Rightarrow \Pi \rightarrow A$
\end{itemize}

We still get stuck at the proof; continue to study a transform lemma
\end{frame}

\begin{frame}{Transform Lemma}
\begin{lemma}[Transform (Sketch)]
If $\Gamma \vdash H \Rightarrow e \Rightarrow A$, then $\Gamma \vdash B \Rightarrow e' \Rightarrow	C$.
\end{lemma}

Finding the ultimate form gives me the feeling that we are actually reducing (back) a typing judgment: \boxed{\Gamma \vdash H \Rightarrow e \Rightarrow A \longmapsto \Gamma \vdash H' \Rightarrow e' \Rightarrow A'}. A small step rule would be
\begin{mathpar}
\inferrule*[lab=unapp]
{ }
{\Gamma \vdash \boxed{e_2} \mapsto H \Rightarrow e_1 \Rightarrow A \rightarrow B
\longmapsto
\Gamma \vdash H \Rightarrow e_1 ~ e_1 \Rightarrow B}
\end{mathpar}

We have to prove that in finite multiple steps, it can be reduced to a normal form $\Gamma \vdash A \Rightarrow e \Rightarrow B$.

Or defining an equational/isomorphic relation, we can reason that any typing judgment is equivalent to a normal form.

\begin{lemma}{Equational typing}
	
\end{lemma}
	
\end{frame}