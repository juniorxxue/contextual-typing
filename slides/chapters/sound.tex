\section{Soundness}

\begin{frame}{Soundness}

I think we can state that for any typing $\Gamma \vdash H \Rightarrow e \Rightarrow A$, we normalise it into $\Gamma \vdash A \Rightarrow e \Rightarrow B$. 

If $A$ is a $Top$, we are doing an inference; 

if $A$ is not a $Top$, we are doing a check.
	
\end{frame}

\section{Completeness}

\begin{frame}{Completeness}

We first draw a relational table:

\begin{tabular}{|c|c|c|}
\hline
    & Dec. & Algo. \\ 
\hline
    Infer & $\Gamma \vdash e \Rightarrow A$ &  $\Gamma \vdash Top \Rightarrow e \Rightarrow A$\\ 
\hline
   Check & $\Gamma \vdash e \Leftarrow A$ & $\Gamma \vdash A \Rightarrow e \Rightarrow A$\\
\hline
\end{tabular}

\begin{lemma}[Completeness (Inf)]
If $\Gamma \vdash e \Rightarrow A$, then $\Gamma \vdash Top \Rightarrow e \Rightarrow A$
\end{lemma}

\begin{lemma}[Completeness (Chk)]
If $\Gamma \vdash e \Leftarrow A$, then $\Gamma \vdash A \Rightarrow e \Rightarrow A$.
\end{lemma}

It can be easily generalised, but we now separate it for justification.

\end{frame}

\begin{frame}{Completeness (Cont.)}

I think the main issues are around the rule T-App2: For example, $(\lambda x . ~x)~1$.
\begin{mathpar}
\inferrule*[lab=T-App2]
{\Gamma \vdash e_1 \Leftarrow A \rightarrow B \\
 \Gamma \vdash e_2 \Rightarrow A}
{\Gamma \vdash e_1~e_2 \Rightarrow B}

\inferrule*[lab=T-App2]
{\Gamma \vdash \lambda x. ~x \Leftarrow Int \rightarrow Top \\
 \Gamma \vdash 1 \Rightarrow Int}
{\Gamma \vdash (\lambda x . ~x)~1 \Rightarrow Top}
\end{mathpar}

The $B$ in rule T-App2 is too relaxed; we may include a \emph{minimal typing}.

\begin{lemma}[Completeness (Inf)]
If $\Gamma \vdash e \Rightarrow A$, then $\Gamma \vdash Top \Rightarrow e \Rightarrow B$ and $\Gamma \vdash B <: A$.
\end{lemma}

\begin{lemma}[Completeness (Chk)]
If $\Gamma \vdash e \Leftarrow A$, then $\Gamma \vdash A \Rightarrow e \Rightarrow B$ and $\Gamma \vdash B <: A$.
\end{lemma}

\end{frame}

%%%%%%%%%% Proof of Completeness %%%%%%%%%%

\begin{frame}{Proof of Completeness}
\begin{theorem}[Completeness]
If $\Gamma \vdash e \Leftrightarrow A$, then $\exists B, \Gamma \vdash (f \Leftrightarrow A) \Rightarrow e \Rightarrow B$ and $\Gamma \vdash B <: A$.
\begin{align*}
f \Leftarrow  A = & ~A \\
f \Rightarrow A  =& ~Top
\end{align*}
\begin{proof}
Induction on $\Gamma \vdash e \Leftrightarrow A$;
\end{proof}
\end{theorem}

Note: we can further simplify the theorem If we have the property that 
\begin{lemma}[Typing Implies Subtyping]
If $\Gamma \vdash A \Rightarrow e \Rightarrow B$, then $\Gamma \vdash B <: A$.
\end{lemma}
\end{frame}

\begin{frame}{Proof of Completeness (Cont.)}
\textbf{case lit:}
\begin{itemize}
	\item given: $\Gamma \vdash n \Rightarrow Int$;
	\item goal: $\exists B, \Gamma \vdash Top \Rightarrow n \Rightarrow B$ and $\Gamma \vdash B <: Int$;
	\item $\exists Int$ and proof is trivial.
\end{itemize}
\noindent\makebox[\linewidth]{\rule{0.9\paperwidth}{0.4pt}}
% ------------------------------------------------------
\textbf{case var:}
\begin{itemize}
	\item given: $\Gamma \vdash x \Rightarrow A$
	\item goal: $\exists B, \Gamma \vdash Top \Rightarrow x \Rightarrow B$ and $\Gamma \vdash B <: A$
	\item $\exists A$ and proof is trivial.
\end{itemize}
\noindent\makebox[\linewidth]{\rule{0.9\paperwidth}{0.4pt}}
% ------------------------------------------------------
\textbf{case lam:}
\begin{itemize}
	\item given: $\Gamma \vdash \lambda x.~e \Leftarrow A \rightarrow B$; that is $\Gamma, x : A \vdash e \Leftarrow B$;
	\item goal: $\exists C, \Gamma \vdash A \rightarrow B \Rightarrow \lambda x.~e  \Rightarrow C$ and $\Gamma \vdash C <: A \rightarrow B$;
	\item by ind hypo, we have $\exists C, \Gamma, x : A \vdash B \Rightarrow e \Rightarrow C$ and $\Gamma, x : A \vdash C <: B$
\end{itemize}
\end{frame}