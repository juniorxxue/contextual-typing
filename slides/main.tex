\documentclass[compress,9pt,aspectratio=169]{beamer}

\usepackage{mathpartir}
\usepackage{amsmath}
\usepackage{mathtools}
\usepackage{listings}

\usepackage{cancel}

\usepackage{comment}

\usetheme[frameno,sans]{Arguelles}

\title{Contextual Type Inference}

\author{Xu Xue}

\begin{document}

\frame[plain]{\titlepage}

\begin{frame} 
\tableofcontents
\end{frame}

\section{Declarative System}

\begin{frame}{Syntax}

\begin{align*}
&\text{Types} \quad\quad &A, B ::=&~ \mathtt{Int} \mid \mathtt{Top} \mid A \rightarrow B\\
&\text{Expressions} \quad \quad &e::=&~ x \mid \lambda x . ~e \mid e_1~e_2 \mid e : A\\
&\text{Contexts} \quad\quad &\Gamma::=&~ . \mid \Gamma, x : A
\end{align*}

\end{frame}

\begin{frame}{Bidirectional Typing: \boxed{\Gamma \vdash e \Leftrightarrow A}}
\begin{mathpar}
\inferrule*[lab=T-Lit]
{ }
{\Gamma \vdash n \Rightarrow Int}

\inferrule*[lab=T-Var]
{x : A \in \Gamma}
{\Gamma \vdash x \Rightarrow A}

\inferrule*[lab=T-Lam]
{\Gamma, x : A \vdash e \Leftarrow B}
{\Gamma \vdash \lambda x.~e \Leftarrow A \rightarrow B}

\inferrule*[lab=T-App1]
{\Gamma \vdash e_1 \Rightarrow A \rightarrow B \\
 \Gamma \vdash e_2 \Leftarrow A}
{\Gamma \vdash e_1~e_2 \Rightarrow B}

\inferrule*[lab=T-App2]
{\Gamma \vdash e_1 \Leftarrow A \rightarrow B \\
 \Gamma \vdash e_2 \Rightarrow A}
{\Gamma \vdash e_1~e_2 \Rightarrow B}

\inferrule*[lab=T-Ann]
{\Gamma \vdash e \Leftarrow A}
{\Gamma \vdash e : A \Rightarrow A}

\inferrule*[lab=T-Sub]
{\Gamma \vdash e \Rightarrow B \\ B <: A}
{\Gamma \vdash e \Leftarrow A}
\end{mathpar}
\end{frame}

\begin{frame}{Subtyping: \boxed{A <: B}}
\begin{mathpar}
\inferrule*[lab=S-Int]
{ }
{Int <: Int}

\inferrule*[lab=S-Top]
{ }
{A <: Top}

\inferrule*[lab=S-Arr]
{C <: A \\ B <: D}
{A \rightarrow B <: C \rightarrow D}
\end{mathpar}
    
\end{frame}

\section{Algo Sys}

\begin{frame}{Syntax \footnote{We reserve $A, B, C, D$ for normal types, and $H$ for hints}}
\begin{align*}
&\text{Hints}\quad\quad &H ::=&~ A \mid \boxed{e} \mapsto H\\
&\text{Normal Types} \quad\quad &A, B ::=&~ \mathtt{Int} \mid \mathtt{Top} \mid A \rightarrow B
\end{align*}

I'm trying to think could it have something like
$$
\boxed{e} \rightarrow A \rightarrow \boxed{e} \rightarrow ... or \quad A \rightarrow \boxed{e} \rightarrow ...
$$

I don't think so, since there're several ways to modify the hint

\begin{itemize}
	\item Lambda, only eliminate the head
	\item Ann, replace the whole hint with a normal type (annotation)
	\item App, chain/append a hole on top of a hint
\end{itemize}


\end{frame}

\begin{frame}{Typing: \boxed{\Gamma \vdash H \Rightarrow e \Rightarrow A}}
\begin{mathpar}
\small
\inferrule*[lab=T-Lit]
{\Gamma \vdash \mathtt{Int} <: H}
{\Gamma \vdash H \Rightarrow i \Rightarrow \mathsf{Int}}

\inferrule*[lab=T-Var]
{x : A \in \Gamma \\
 \Gamma \vdash A <: H}
{\Gamma \vdash H \Rightarrow x \Rightarrow A}

\inferrule*[lab=T-App]
{\Gamma \vdash \boxed{e_2} \mapsto H \Rightarrow e_1 \Rightarrow A \rightarrow B}
{\Gamma \vdash H \Rightarrow e_1~e_2 \Rightarrow B}

\inferrule*[lab=T-Ann]
{\Gamma \vdash A \Rightarrow e \Rightarrow B \\
 \Gamma \vdash A <: H}
{\Gamma \vdash H \Rightarrow e : A \Rightarrow A}

\inferrule*[lab=T-Lam1]
{\Gamma \vdash \mathtt{Top} \Rightarrow e_1 \Rightarrow A \\
 \Gamma, x : A \vdash H \Rightarrow e \Rightarrow B}
{\Gamma \vdash \boxed{e_1} \mapsto H \Rightarrow \lambda x.~e \Rightarrow A \rightarrow B}

\inferrule*[lab=T-Lam2]
{\Gamma, x : A \vdash B \Rightarrow e \Rightarrow C}
{\Gamma \vdash A \rightarrow B \Rightarrow \lambda x.~e \Rightarrow A \rightarrow C}
\end{mathpar} 
\end{frame}

\begin{frame}{Subtyping: \boxed{\Gamma \vdash A <: H}}
\begin{mathpar}
\inferrule*[lab=S-Refl]	
{ }
{\Gamma \vdash \mathtt{Int} <: \mathtt{Int}}

\inferrule*[lab=S-Top]
{ }
{\Gamma \vdash A <: \mathtt{Top}}

\inferrule*[lab=S-Arr]
{\Gamma \vdash C <: A \\
 \Gamma \vdash B <: D}
{\Gamma \vdash A \rightarrow B <: C \rightarrow D}

\inferrule*[lab=S-Chain]
{\Gamma \vdash A \Rightarrow e \Rightarrow C	 \\
 \Gamma \vdash B <: H}
{\Gamma \vdash A \rightarrow B <: \boxed{e} \mapsto H}
\end{mathpar}    
\end{frame}

\begin{frame}{Split: \boxed{(H , A) \rightarrowtail (\bar{e} , T, A')}\footnote{Observation: we know that $\Gamma \vdash \boxed{e} \mapsto H \Rightarrow e \Rightarrow A \rightarrow B$ is always true, thus it covers all cases.}}
\begin{mathpar}
\inferrule*[lab=none]	
{ }
{(T, A) \rightarrowtail ([], T, A)}

\inferrule*[lab=have]
{(H, B) \rightarrowtail (\bar{e}, T', B')}
{(\boxed{e} \mapsto H,  (A \rightarrow B)) \rightarrowtail (e :: \bar{e}, T', B')}
\end{mathpar}
\end{frame}

\begin{frame}{Applications: $e \triangleright \bar{e} = e'$}

\begin{mathpar}
\inferrule*[lab=empty]
{ }
{e \triangleright [] = e}

\inferrule*[lab=cons]
{ }
{e_1 \triangleright (e_2 :: \bar{e}) = (e_1~e_2) :: \bar{e}}

\end{mathpar}
\end{frame}

\begin{frame}{Transform Lemma}
\begin{lemma}[Transform]
If $\Gamma \vdash H \Rightarrow e \Rightarrow A$ and $(H, A) \rightarrowtail (\bar{e}, T, A')$, then $\Gamma \vdash T \Rightarrow e \triangleright \bar{e} \Rightarrow A'$.
\end{lemma}

\begin{lemma}[Split]
If $\Gamma \vdash H \Rightarrow e \Rightarrow A$, then $\exists (\bar{e}, T, A'), (H , A) \rightarrowtail (\bar{e} , T, A')$.
\end{lemma}

\end{frame}

\begin{frame}{Properties of Algo.}
\begin{lemma}[Typing implies Subtyping]
If $\Gamma \vdash H \Rightarrow e \Rightarrow A$ then $\Gamma \vdash A <: H$.
\end{lemma}

\begin{lemma}[Hint Self]
If $\Gamma \vdash H \Rightarrow e \Rightarrow A$, then $\Gamma \vdash A \Rightarrow e \Rightarrow A$.
\end{lemma}
\end{frame}

\section{Soundness}

\begin{frame}{Soundness}

I think we can state that for any typing $\Gamma \vdash H \Rightarrow e \Rightarrow A$, we normalise it into $\Gamma \vdash A \Rightarrow e \Rightarrow B$.

If $A$ is a $Top$, we are doing an inference;

if $A$ is not a $Top$, we are doing a check.

\end{frame}

\begin{frame}{Naive Soundness}
\begin{theorem}[Soundness (Chk)]
If $\Gamma \vdash A \Rightarrow e \Rightarrow B$, then $\Gamma \vdash e \Leftarrow A$.
\end{theorem}

other cases are easy, \textbf{case app:}
\begin{itemize}
	\item given: $\Gamma \vdash \boxed{e_2} \mapsto A \Rightarrow e_1 \Rightarrow B \rightarrow C$;
	\item goal: $\Gamma \vdash e_1 ~ e_2 \Leftarrow A$
\end{itemize}

\begin{theorem}[Soundness (Inf)]
If $\Gamma \vdash Top \Rightarrow e \Rightarrow A$, then $\Gamma \vdash e \Rightarrow A$.
\end{theorem}

other cases are easy, \textbf{case ann} is dependent on the above theorem;

\textbf{case app:}

\begin{itemize}
	\item given: $\Gamma \vdash \boxed{e_2} \mapsto Top \Rightarrow e_1 \Rightarrow A \rightarrow B$;
	\item goal: $\Gamma \vdash e_1 ~ e_2 \Rightarrow B$
\end{itemize}

\end{frame}

% #region [rgba(0,196,25,0.15)]
\section{Completeness}

\begin{frame}{Completeness}

We first draw a relational table:

\begin{tabular}{|c|c|c|}
\hline
    & Dec. & Algo. \\
\hline
    Infer & $\Gamma \vdash e \Rightarrow A$ &  $\Gamma \vdash Top \Rightarrow e \Rightarrow A$\\
\hline
   Check & $\Gamma \vdash e \Leftarrow A$ & $\Gamma \vdash A \Rightarrow e \Rightarrow A$\\
\hline
\end{tabular}

\begin{lemma}[Completeness (Inf)]
If $\Gamma \vdash e \Rightarrow A$, then $\Gamma \vdash Top \Rightarrow e \Rightarrow A$
\end{lemma}

\begin{lemma}[Completeness (Chk)]
If $\Gamma \vdash e \Leftarrow A$, then $\Gamma \vdash A \Rightarrow e \Rightarrow A$.
\end{lemma}

It can be easily generalised, but we now separate it for justification.

\end{frame}

\begin{frame}{Completeness (Cont.)}

I think the main issues are around the rule T-App2: For example, $(\lambda x . ~x)~1$.
\begin{mathpar}
\inferrule*[lab=T-App2]
{\Gamma \vdash e_1 \Leftarrow A \rightarrow B \\
 \Gamma \vdash e_2 \Rightarrow A}
{\Gamma \vdash e_1~e_2 \Rightarrow B}

\inferrule*[lab=T-App2]
{\Gamma \vdash \lambda x. ~x \Leftarrow Int \rightarrow Top \\
 \Gamma \vdash 1 \Rightarrow Int}
{\Gamma \vdash (\lambda x . ~x)~1 \Rightarrow Top}
\end{mathpar}

The $B$ in rule T-App2 is too relaxed; we may include a \emph{minimal typing}.

\begin{lemma}[Completeness (Inf)]
If $\Gamma \vdash e \Rightarrow A$, then $\Gamma \vdash Top \Rightarrow e \Rightarrow B$ and $\Gamma \vdash B <: A$.
\end{lemma}

\begin{lemma}[Completeness (Chk)]
If $\Gamma \vdash e \Leftarrow A$, then $\Gamma \vdash A \Rightarrow e \Rightarrow B$ and $\Gamma \vdash B <: A$.
\end{lemma}

\end{frame}
%end region

% #region [rgba(241,196,15,0.15)]
%%%%%%%%%% Proof of Completeness %%%%%%%%%%

\begin{frame}{Proof of Completeness}
\begin{theorem}[Completeness]
If $\Gamma \vdash e \Leftrightarrow A$, then $\exists B, \Gamma \vdash (f \Leftrightarrow A) \Rightarrow e \Rightarrow B$ and $\Gamma \vdash B <: A$.
\begin{align*}
f \Leftarrow  A = & ~A \\
f \Rightarrow A  =& ~Top
\end{align*}
\begin{proof}
Induction on $\Gamma \vdash e \Leftrightarrow A$;
\end{proof}
\end{theorem}

Note: we can further simplify the theorem (check condition) If we have the property that
\begin{lemma}[Typing Implies Subtyping]
If $\Gamma \vdash A \Rightarrow e \Rightarrow B$, then $\Gamma \vdash B <: A$.
\end{lemma}
\end{frame}

\begin{frame}{Proof of Completeness (Cont.)}
\textbf{case lit:}
\begin{itemize}
	\item given: $\Gamma \vdash n \Rightarrow Int$;
	\item goal: $\exists B, \Gamma \vdash Top \Rightarrow n \Rightarrow B$ and $\Gamma \vdash B <: Int$;
	\item $\exists Int$ and proof is trivial.
\end{itemize}
\noindent\makebox[\linewidth]{\rule{0.9\paperwidth}{0.4pt}}
% ------------------------------------------------------
\textbf{case var:}
\begin{itemize}
	\item given: $\Gamma \vdash x \Rightarrow A$
	\item goal: $\exists B, \Gamma \vdash Top \Rightarrow x \Rightarrow B$ and $\Gamma \vdash B <: A$
	\item $\exists A$ and proof is trivial.
\end{itemize}
\noindent\makebox[\linewidth]{\rule{0.9\paperwidth}{0.4pt}}
% ------------------------------------------------------
\textbf{case lam:}
\begin{itemize}
	\item given: $\Gamma \vdash \lambda x.~e \Leftarrow A \rightarrow B$; that is $\Gamma, x : A \vdash e \Leftarrow B$;
	\item goal: $\exists C, \Gamma \vdash A \rightarrow B \Rightarrow \lambda x.~e  \Rightarrow C$ and $\Gamma \vdash C <: A \rightarrow B$;
	\item by ind hypo, we have $\exists C, \Gamma, x : A \vdash B \Rightarrow e \Rightarrow C$ and $\Gamma, x : A \vdash C <: B$;
\end{itemize}
\end{frame}

\begin{frame}{Proof of Completeness (Cont.)}
	\begin{itemize}
		\item that is we have $\Gamma, x : A \vdash B \Rightarrow e \Rightarrow B'$ and $\Gamma, x : A \vdash B' <: B$;
		\item $\exists A \rightarrow B'$ and proof is obvious.
	\end{itemize}
\noindent\makebox[\linewidth]{\rule{0.9\paperwidth}{0.4pt}}
% ------------------------------------------------------
\textbf{case app1:}
\begin{itemize}
	\item given: $\Gamma \vdash e_1~e_2 \Rightarrow A$, that is $\Gamma \vdash e_1 \Rightarrow A \rightarrow B$ and $\Gamma \vdash e_2 \Leftarrow A$;
	\item goal: $\exists T, \Gamma \vdash Top \Rightarrow e_1~e_2 \Rightarrow T$ and $\Gamma \vdash T <: B$;
	\item that is we want to prove $\Gamma \vdash \boxed{e_2} \mapsto Top \Rightarrow e_1 \Rightarrow \Delta \rightarrow T$;
	\item  by ind hypo we know what $e_1$ is inferrable $\Gamma \vdash Top \Rightarrow e_1 \Rightarrow A' \rightarrow B'$ and $e_2$ is checkable $\Gamma \vdash A \Rightarrow e_2 \Rightarrow C$;
	\item we want a new lemma (discussed before)
\end{itemize}
\noindent\makebox[\linewidth]{\rule{0.9\paperwidth}{0.4pt}}
% ------------------------------------------------------
\begin{lemma}
If $e_1$ is inferrable to a arrow type (normalize it, and the hint is a $Top$), then pending an argument (checked by input type) doesn't affect its inferred type.
\end{lemma}

\end{frame}

\begin{frame}{Proof of Completeness (Cont.)}
\textbf{case app2:}
\begin{itemize}
	\item given: $\Gamma \vdash e_1~e_2 \Rightarrow A$, that is $\Gamma \vdash e_1 \Leftarrow A \rightarrow B$ and $\Gamma \vdash e_2 \Rightarrow A$;
	\item goal: $\exists T, \Gamma \vdash Top \Rightarrow e_1~e_2 \Rightarrow T$ and $\Gamma \vdash T <: B$;
		\item that is we want to prove $\Gamma \vdash \boxed{e_2} \mapsto Top \Rightarrow e_1 \Rightarrow \Delta \rightarrow T$;
	\item  by ind hypo we know what $e_1$ is checkable $\Gamma \vdash A \rightarrow B \Rightarrow e_1 \Rightarrow A' \rightarrow B'$ and $e_2$ is inferrable $\Gamma \vdash Top \Rightarrow e_2 \Rightarrow A''$;
\end{itemize}
\noindent\makebox[\linewidth]{\rule{0.9\paperwidth}{0.4pt}}
% ------------------------------------------------------
\textbf{case ann:}
\noindent\makebox[\linewidth]{\rule{0.9\paperwidth}{0.4pt}}
% ------------------------------------------------------
\textbf{case sub:}
\begin{itemize}
	\item given: $\Gamma \vdash e \Rightarrow B$ and $B <: A$;
	\item goal: $\exists C, \Gamma \vdash A \Rightarrow e \Rightarrow C$ and $\Gamma \vdash C <: A$;
	\item by ind hypo, we have $\Gamma \vdash Top \Rightarrow e \Rightarrow B'$ and $\Gamma \vdash B' <: B$;
	\item and we need a lemma (algo sys) to move from infer to check.
\end{itemize}
\end{frame}

% #end region

\section{Cases}

\begin{frame}{$(\lambda x. ~x)~1$}
\begin{mathpar}
\inferrule*[Right=T-App]
{
\inferrule*[Right=T-Lam1]
{. \vdash Top \Rightarrow 1 \Rightarrow Int \\ x : Int \vdash Top \Rightarrow x \Rightarrow Int}
{. \vdash \boxed{1} \rightarrow Top \Rightarrow \lambda x. ~x \Rightarrow Int \rightarrow Int}
}
{. \vdash Top \Rightarrow (\lambda x. ~x)~1 \Rightarrow Int}
\end{mathpar}
\end{frame}

\begin{frame}{$(\lambda f . ~ f ~ 1) : ((Int \rightarrow Int) \rightarrow Int)~ (\lambda x. ~x)$}

\begin{mathpar}
\inferrule*[Right=T-App]
%%%%%% T-App Premise
{
\inferrule*[Right=T-Ann]
{
\inferrule*[Right=T-Lam2]{ }{. \vdash (Int \rightarrow Int) \rightarrow Int \Rightarrow \lambda f . ~ f ~ 1 \Rightarrow (Int \rightarrow Int) \rightarrow Int} \quad \quad \quad \quad
\inferrule*[Right=S-Arr]{ }{. \vdash (Int \rightarrow Int) \rightarrow Int <: \boxed{\lambda x. ~x} \rightarrow Top}
}
{. \vdash \boxed{\lambda x. ~x} \rightarrow Top \Rightarrow (\lambda f . ~ f ~ 1) : ((Int \rightarrow Int) \rightarrow Int) \Rightarrow (Int \rightarrow Int) \rightarrow Int}
}
%%%%%% T-App Conclusion
{. \vdash Top \Rightarrow (\lambda f . ~ f ~ 1) : ((Int \rightarrow Int) \rightarrow Int)~ (\lambda x. ~x) \Rightarrow Int}
\end{mathpar}

For the second premise, we have

\begin{mathpar}
\inferrule*[Right=S-Arr]
{
\inferrule*[Right=S-Tele]
{. \vdash Int \rightarrow Int \Rightarrow \lambda x. ~x \Rightarrow Int \rightarrow Int}
{ . \vdash \boxed{\lambda x. ~x} <: Int \rightarrow Int} \\
. \vdash Int <: Top
}
{. \vdash (Int \rightarrow Int) \rightarrow Int <: \boxed{\lambda x. ~x} \rightarrow Top}
\end{mathpar}
	
\end{frame}

\end{document}
