\documentclass[compress,12pt,aspectratio=169]{beamer}

\usepackage{mathpartir}
\usepackage{amsmath}
\usepackage{mathtools}
\usepackage{listings}

\usepackage{cancel}

\usepackage{comment}

\usetheme[frameno,sans]{Arguelles}

\title{Contextual Type Inference}

\author{Xu Xue}

\begin{document}

\frame[plain]{\titlepage}

\section{Traditional}

\begin{frame}{Dec. Syntax}

\begin{align*}
&\text{Types} \quad\quad &A, B ::=&~ \mathtt{Int} \mid \mathtt{Top} \mid A \rightarrow B\\
&\text{Expressions} \quad \quad &e::=&~ x \mid \lambda x . ~e \mid e_1~e_2 \mid e : A\\
&\text{Contexts} \quad\quad &\Gamma::=&~ . \mid \Gamma, x : A
\end{align*}

\end{frame}

\begin{frame}{Dec. Typing: \boxed{\Gamma \vdash e \Leftrightarrow A}}
\begin{mathpar}
\inferrule*[lab=T-Lit]
{ }
{\Gamma \vdash n \Rightarrow Int}

\inferrule*[lab=T-Var]
{x : A \in \Gamma}
{\Gamma \vdash x \Rightarrow A}

\inferrule*[lab=T-Lam]
{\Gamma, x : A \vdash e \Leftarrow B}
{\Gamma \vdash \lambda x.~e \Leftarrow A \rightarrow B}

\inferrule*[lab=T-App1]
{\Gamma \vdash e_1 \Rightarrow A \rightarrow B \\
 \Gamma \vdash e_2 \Leftarrow A}
{\Gamma \vdash e_1~e_2 \Rightarrow B}

\inferrule*[lab=T-App2]
{\Gamma \vdash e_1 \Leftarrow A \rightarrow B \\
 \Gamma \vdash e_2 \Rightarrow A}
{\Gamma \vdash e_1~e_2 \Leftarrow B}

\inferrule*[lab=T-Ann]
{\Gamma \vdash e \Leftarrow A}
{\Gamma \vdash e : A \Rightarrow A}

\inferrule*[lab=T-Sub]
{\Gamma \vdash e \Rightarrow B \\ B <: A}
{\Gamma \vdash e \Leftarrow A}
\end{mathpar}
\end{frame}

\begin{frame}{Dec. Subtyping: \boxed{A <: B}}
\begin{mathpar}
\inferrule*[lab=S-Int]
{ }
{Int <: Int}

\inferrule*[lab=S-Top]
{ }
{A <: Top}

\inferrule*[lab=S-Arr]
{C <: A \\ B <: D}
{A \rightarrow B <: C \rightarrow D}
\end{mathpar}

\end{frame}

\begin{frame}{Algo. Syntax}

\begin{align*}
&\text{Hints}\quad\quad &H ::=&~ A \mid \boxed{e} \mapsto H\\
&\text{Normal Types} \quad\quad &A, B ::=&~ \mathtt{Int} \mid \mathtt{Top} \mid A \rightarrow B
\end{align*}

\end{frame}


\begin{frame}{Algo. Typing: \boxed{\Gamma \vdash H \Rightarrow e \Rightarrow A}}
\begin{mathpar}
\small
\inferrule*[lab=T-Lit]
{\Gamma \vdash \mathtt{Int} <: H}
{\Gamma \vdash H \Rightarrow i \Rightarrow \mathsf{Int}}

\inferrule*[lab=T-Var]
{x : A \in \Gamma \\
 \Gamma \vdash A <: H}
{\Gamma \vdash H \Rightarrow x \Rightarrow A}

\inferrule*[lab=T-App]
{\Gamma \vdash \boxed{e_2} \mapsto H \Rightarrow e_1 \Rightarrow A \rightarrow B
\\ \Gamma \vdash B <: H}
{\Gamma \vdash H \Rightarrow e_1~e_2 \Rightarrow B}

\inferrule*[lab=T-Ann]
{\Gamma \vdash A \Rightarrow e \Rightarrow B \\
 \Gamma \vdash A <: H}
{\Gamma \vdash H \Rightarrow e : A \Rightarrow A}

%\inferrule*[lab=T-Lam1]
%{\Gamma \vdash \mathtt{Top} \Rightarrow e_1 \Rightarrow A \\
% \Gamma, x : A \vdash H \Rightarrow e \Rightarrow B}
%{\Gamma \vdash \boxed{e_1} \mapsto H \Rightarrow \lambda x.~e \Rightarrow A \rightarrow B}

\inferrule*[lab=T-Lam]
{\Gamma, x : A \vdash B \Rightarrow e \Rightarrow C}
{\Gamma \vdash A \rightarrow B \Rightarrow \lambda x.~e \Rightarrow A \rightarrow C}
\end{mathpar} 
\end{frame}

\begin{frame}{Algo. Subtyping: \boxed{\Gamma \vdash A <: H}}
\begin{mathpar}
\inferrule*[lab=S-Refl]	
{ }
{\Gamma \vdash \mathtt{Int} <: \mathtt{Int}}

\inferrule*[lab=S-Top]
{ }
{\Gamma \vdash A <: \mathtt{Top}}

\inferrule*[lab=S-Arr]
{\Gamma \vdash C <: A \\
 \Gamma \vdash B <: D}
{\Gamma \vdash A \rightarrow B <: C \rightarrow D}

\inferrule*[lab=S-Chain]
{\Gamma \vdash A \Rightarrow e \Rightarrow C	 \\
 \Gamma \vdash B <: H}
{\Gamma \vdash A \rightarrow B <: \boxed{e} \mapsto H}
\end{mathpar}    
\end{frame}

\section{Soundness \& Completeness}

\begin{frame}{Soundness}
	\begin{lemma}[Soundness]
		If $\Gamma \vdash H \Rightarrow e \Rightarrow A$ and  $(H, A) \rightarrowtail (\bar{e}, T, \bar{A}, A')$, then $\Gamma \vdash e \vartriangleright \bar{e} \Leftarrow A'$.
	\end{lemma}
	
Split: \boxed{(H , A) \rightarrowtail (\bar{e} ,T, \bar{A}, A')}
\begin{mathpar}
\inferrule*[lab=none]	
{ }
{(T, A) \rightarrowtail ([], T, [], A)}

\inferrule*[lab=have]
{(H, B) \rightarrowtail (\bar{e}, T', \bar{B}, B')}
{(\boxed{e} \mapsto H,  (A \rightarrow B)) \rightarrowtail (e :: \bar{e}, T', A::\bar{B}, B')}
\end{mathpar}

Applications: \boxed{e \triangleright \bar{e} = e'}
\begin{mathpar}
\inferrule*[lab=empty]
{ }
{e \triangleright [] = e}

\inferrule*[lab=cons]
{ }
{e_1 \triangleright (e_2 :: \bar{e}) = (e_1~e_2) :: \bar{e}}

\end{mathpar}
		
\end{frame}

\begin{frame}{Completeness}
\begin{lemma}[Completeness]
If $\Gamma \vdash e \Leftarrow A$, then $\exists B, \Gamma \vdash A \Rightarrow e \Rightarrow B$.

If $\Gamma \vdash e \Rightarrow A$, then $\Gamma \vdash Top \Rightarrow e \Rightarrow A$.
\end{lemma}
\end{frame}

\section{w/ Counters}

\begin{frame}{w/ Counters: $\Gamma \vdash_j e \Leftrightarrow A$}
\begin{mathpar}
\inferrule*[lab=T-Lit]
{ }
{\Gamma \vdash_j  n \Rightarrow Int}

\inferrule*[lab=T-Var]
{x : A \in \Gamma}
{\Gamma \vdash_j x \Rightarrow A}

\inferrule*[lab=T-Lam]
{\Gamma, x : A \vdash_{j-} e \Leftarrow B}
{\Gamma \vdash_j \lambda x.~e \Leftarrow A \rightarrow B}

\inferrule*[lab=T-App1]
{\Gamma \vdash e_1 \Rightarrow A \rightarrow B \\
 \Gamma \vdash e_2 \Leftarrow A}
{\Gamma \vdash e_1~e_2 \Rightarrow B}

\inferrule*[lab=T-App2]
{\Gamma \vdash_{j+} e_1 \Leftarrow A \rightarrow B \\
 \Gamma \vdash_0 e_2 \Rightarrow A}
{\Gamma \vdash_j e_1~e_2 \Rightarrow B}

\inferrule*[lab=T-Ann]
{\Gamma \vdash_{\circ} e \Leftarrow A}
{\Gamma \vdash_j e : A \Rightarrow A}

\inferrule*[lab=T-Sub]
{\Gamma \vdash_j e \Rightarrow B \\ B <: A}
{\Gamma \vdash_j e \Leftarrow A}
\end{mathpar}
\end{frame}

\end{document}
