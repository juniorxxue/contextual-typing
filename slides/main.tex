\documentclass[compress,9pt,aspectratio=169]{beamer}

\usepackage{mathpartir}
\usepackage{amsmath}
\usepackage{mathtools}

\usepackage{cancel}

\usepackage{comment}

\usetheme[frameno,sans]{Arguelles}

\title{Contextual Type Inference}

\author{Xu Xue}

\begin{document}

\frame[plain]{\titlepage}

\begin{frame} 
\tableofcontents
\end{frame}

\section{Declarative System}

\begin{frame}{Syntax}

\begin{align*}
&\text{Types} \quad\quad &A, B ::=&~ \mathtt{Int} \mid \mathtt{Top} \mid A \rightarrow B\\
&\text{Expressions} \quad \quad &e::=&~ x \mid \lambda x . ~e \mid e_1~e_2 \mid e : A\\
&\text{Contexts} \quad\quad &\Gamma::=&~ . \mid \Gamma, x : A
\end{align*}

\end{frame}

\begin{frame}{Bidirectional Typing: \boxed{\Gamma \vdash e \Leftrightarrow A}}
\begin{mathpar}
\inferrule*[lab=T-Lit]
{ }
{\Gamma \vdash n \Rightarrow Int}

\inferrule*[lab=T-Var]
{x : A \in \Gamma}
{\Gamma \vdash x \Rightarrow A}

\inferrule*[lab=T-Lam]
{\Gamma, x : A \vdash e \Leftarrow B}
{\Gamma \vdash \lambda x.~e \Leftarrow A \rightarrow B}

\inferrule*[lab=T-App1]
{\Gamma \vdash e_1 \Rightarrow A \rightarrow B \\
 \Gamma \vdash e_2 \Leftarrow A}
{\Gamma \vdash e_1~e_2 \Rightarrow B}

\inferrule*[lab=T-App2]
{\Gamma \vdash e_1 \Leftarrow A \rightarrow B \\
 \Gamma \vdash e_2 \Rightarrow A}
{\Gamma \vdash e_1~e_2 \Rightarrow B}

\inferrule*[lab=T-Ann]
{\Gamma \vdash e \Leftarrow A}
{\Gamma \vdash e : A \Rightarrow A}

\inferrule*[lab=T-Sub]
{\Gamma \vdash e \Rightarrow B \\ B <: A}
{\Gamma \vdash e \Leftarrow A}
\end{mathpar}
\end{frame}

\begin{frame}{Subtyping: \boxed{A <: B}}
\begin{mathpar}
\inferrule*[lab=S-Int]
{ }
{Int <: Int}

\inferrule*[lab=S-Top]
{ }
{A <: Top}

\inferrule*[lab=S-Arr]
{C <: A \\ B <: D}
{A \rightarrow B <: C \rightarrow D}
\end{mathpar}
    
\end{frame}

\section{Algo Sys}

\begin{frame}{Syntax \footnote{We reserve $A, B, C, D$ for normal types, and $H$ for hints}}
\begin{align*}
&\text{Hints}\quad\quad &H ::=&~ A \mid \boxed{e} \mapsto H\\
&\text{Normal Types} \quad\quad &A, B ::=&~ \mathtt{Int} \mid \mathtt{Top} \mid A \rightarrow B
\end{align*}

I'm trying to think could it have something like
$$
\boxed{e} \rightarrow A \rightarrow \boxed{e} \rightarrow ... or \quad A \rightarrow \boxed{e} \rightarrow ...
$$

I don't think so, since there're several ways to modify the hint

\begin{itemize}
	\item Lambda, only eliminate the head
	\item Ann, replace the whole hint with a normal type (annotation)
	\item App, chain/append a hole on top of a hint
\end{itemize}


\end{frame}

\begin{frame}{Typing: \boxed{\Gamma \vdash H \Rightarrow e \Rightarrow A}}
\begin{mathpar}
\small
\inferrule*[lab=T-Lit]
{\Gamma \vdash \mathtt{Int} <: H}
{\Gamma \vdash H \Rightarrow i \Rightarrow \mathsf{Int}}

\inferrule*[lab=T-Var]
{x : A \in \Gamma \\
 \Gamma \vdash A <: H}
{\Gamma \vdash H \Rightarrow x \Rightarrow A}

\inferrule*[lab=T-App]
{\Gamma \vdash \boxed{e_2} \mapsto H \Rightarrow e_1 \Rightarrow A \rightarrow B}
{\Gamma \vdash H \Rightarrow e_1~e_2 \Rightarrow B}

\inferrule*[lab=T-Ann]
{\Gamma \vdash A \Rightarrow e \Rightarrow B \\
 \Gamma \vdash A <: H}
{\Gamma \vdash H \Rightarrow e : A \Rightarrow A}

\inferrule*[lab=T-Lam1]
{\Gamma \vdash \mathtt{Top} \Rightarrow e_1 \Rightarrow A \\
 \Gamma, x : A \vdash H \Rightarrow e \Rightarrow B}
{\Gamma \vdash \boxed{e_1} \mapsto H \Rightarrow \lambda x.~e \Rightarrow A \rightarrow B}

\inferrule*[lab=T-Lam2]
{\Gamma, x : A \vdash B \Rightarrow e \Rightarrow C}
{\Gamma \vdash A \rightarrow B \Rightarrow \lambda x.~e \Rightarrow A \rightarrow C}
\end{mathpar}    
\end{frame}

\begin{frame}{Subtyping: \boxed{\Gamma \vdash A <: H}}
\begin{mathpar}
\inferrule*[lab=S-Refl]	
{ }
{\Gamma \vdash \mathtt{Int} <: \mathtt{Int}}

\inferrule*[lab=S-Top]
{ }
{\Gamma \vdash A <: \mathtt{Top}}

\inferrule*[lab=S-Arr]
{\Gamma \vdash C <: A \\
 \Gamma \vdash B <: D}
{\Gamma \vdash A \rightarrow B <: C \rightarrow D}

\inferrule*[lab=S-Chain]
{\Gamma \vdash A \Rightarrow e \Rightarrow C	 \\
 \Gamma \vdash B <: H}
{\Gamma \vdash A \rightarrow B <: \boxed{e} \mapsto H}
\end{mathpar}    
\end{frame}

\begin{frame}{Properties}
\begin{lemma}[$\vdash$ to $<:$]
If $\Gamma \vdash H \Rightarrow e \Rightarrow A$ then $\Gamma \vdash A <: H$.
\end{lemma}
\begin{proof}
Induction on $\Gamma \vdash H \Rightarrow e \Rightarrow A$; other cases are trivial, we only focus on first case of lambda
\end{proof}

\begin{itemize}
	\item \textbf{case lam1:} goal: $\Gamma \vdash A \rightarrow B <: \boxed{e_1} \mapsto H$, given $\Gamma \vdash Top \Rightarrow e_1 \Rightarrow A$ and $\Gamma, x : A \vdash H \Rightarrow e \Rightarrow B$;
	\item that is to prove $\Gamma \vdash A \Rightarrow e \Rightarrow A$ and $\Gamma \vdash B <: H$;
	\item with ind hypo, we can have $\Gamma, x : A \vdash B <: H$;
\end{itemize}

\noindent\makebox[\linewidth]{\rule{\paperwidth}{0.4pt}}

\begin{lemma}[No hint and full hint]
If $\Gamma \vdash Top \Rightarrow e \Rightarrow A$, then $\Gamma \vdash A \Rightarrow e \Rightarrow A$.
\end{lemma}

\end{frame}

\begin{frame}{Proof of No hint and full hint}
\begin{lemma}[No hint and full hint]
If $\Gamma \vdash Top \Rightarrow e \Rightarrow A$, then $\Gamma \vdash A \Rightarrow e \Rightarrow A$.
\end{lemma}
The tricky case is app:
\begin{itemize}
	\item goal: $\Gamma \vdash A \Rightarrow e_1~e_2 \Rightarrow A$, given $\Gamma \vdash Top \Rightarrow e_1~e_2 \Rightarrow A$;
	\item that is to prove $\Gamma \vdash \boxed{e_2} \mapsto A \Rightarrow e_1 \Rightarrow \Delta \rightarrow A$;
	\item and we have $\Gamma \vdash \boxed{e_2} \rightarrow Top \Rightarrow e_1 \Rightarrow \Pi \rightarrow A$;
\end{itemize}
\end{frame}

\begin{frame}{Proof of No hint and full hint (Cont.)}
\begin{lemma}[No hint and full hint (Gen)]
If $\Gamma \vdash H \Rightarrow e \Rightarrow A$, then $\Gamma \vdash A \Rightarrow e \Rightarrow A$.
\end{lemma}

We generalize it to the max; the app case will be
\begin{itemize}
	\item goal: $\Gamma \vdash A \Rightarrow e_1~e_2 \Rightarrow A$, given $\Gamma \vdash H \Rightarrow e_1~e_2 \Rightarrow A$;
	\item that is to prove $\Gamma \vdash \boxed{e_2} \mapsto A \Rightarrow e_1 \Rightarrow \Delta \rightarrow A$;
	\item and we have $\Gamma \vdash \boxed{e_2} \rightarrow H \Rightarrow e_1 \Rightarrow \Pi \rightarrow A$;
	\item by ind hypo, we have $\Gamma \vdash \Pi \rightarrow A \Rightarrow e_1 \Rightarrow \Pi \rightarrow A$
\end{itemize}

We still get stuck at the proof; continue to study a transform lemma
\end{frame}

\begin{frame}[Properties]
We state a special subtyping relation between holes (read as "more informative").


\end{frame}
\section{Soundness}

\begin{frame}{Soundness}

	\begin{lemma}[Soundness]
		If $\Gamma \vdash H \Rightarrow e \Rightarrow A$ and  $(H, A) \rightarrowtail (\bar{e}, T, A')$, then $\Gamma \vdash e \vartriangleright \bar{e} \Leftarrow A'$.
	\end{lemma}

	\begin{proof}
		Induction on \cancel{algo typing} \cancel{split} algo typing.
	\end{proof}

	\noindent\makebox[\linewidth]{\rule{0.9\paperwidth}{0.4pt}}
	% ------------------------------------------------------

	Justification of the statement: $\Leftarrow$ only accounts for the case $\Gamma \vdash Int \rightarrow Int \Rightarrow \lambda x. ~x \Rightarrow Int \rightarrow Int$. (it is only one of the base case, the most other cases is proved by $\Rightarrow$ by using T-Sub and reflexivity.)

\end{frame}

\begin{frame}{Proof of Soundness}

	\textbf{case lit:} $\Gamma \vdash H \Rightarrow n \Rightarrow Int$ and $(H, Int) \rightarrowtail ([], H, Int)$
	\begin{itemize}
		\item goal: $\Gamma \vdash e \Leftarrow Int$;
		\item by T-Sub and T-Lit
	\end{itemize}

\end{frame}

\begin{frame}{Proof of Soundness (Cont.)}
	\textbf{case var:} $\Gamma \vdash H \Rightarrow x \Rightarrow A$ and $(H, A) \rightarrowtail (\bar{e}, T, A')$

	\begin{itemize}
		\item goal: $\Gamma \vdash x \triangleright \bar{e} \Leftarrow A'$;
		\item given: $x : A \in \Gamma$ and $\Gamma \vdash A <: H$;
		\item by T-Var, we then know $\Gamma \vdash x \Rightarrow A$;
		\item further proved by the following proposed lemmas
	\end{itemize}

	\noindent\makebox[\linewidth]{\rule{0.9\paperwidth}{0.4pt}}
	% ------------------------------------------------------

	\begin{lemma}[Function Type Elimination]
		If $\Gamma \vdash e \Rightarrow A$ and $A \rightharpoonup ^n A'$ and $\bar{e}$ is checkable by its arguments, then $\Gamma \vdash e \triangleright ^{n} \bar{e} \Rightarrow A'$ .
	\end{lemma}

	\begin{lemma}[Checkable Arguments]
		If $\Gamma \vdash A <: H$, then each hole in $H$ is checkable by the corresponding input type of $A$.
	\end{lemma}

\end{frame}

\begin{frame}{Proof of Soundness (Cont.)}
	\textbf{case app:} $\Gamma \vdash H \Rightarrow e_1 ~ e_2 \Rightarrow A$ and $(H, A) \rightarrowtail (\bar{e}, T, A')$

	\begin{itemize}
		\item goal: $\Gamma \vdash e_1 ~ e_2 \triangleright \bar{e} \Leftarrow A'$;
		\item given: $\Gamma \vdash \boxed{e_2} \mapsto H \Rightarrow e_1 \Rightarrow \Delta \rightarrow A$;
		\item which is to prove $\Gamma \vdash e_1 \triangleright (e_2 :: \bar{e}) \Leftarrow A'$;
		\item by ind hypo, we can prove the goal if we have $(e_2 \mapsto H, \Delta \rightarrow A) \rightarrowtail (\bar{e}, T, A')$;
		\item proved by rule \textsc{Have}.
	\end{itemize}

\end{frame}

\begin{frame}{Proof of Soundness (Cont.)}
	\textbf{case ann:} $\Gamma \vdash H \Rightarrow e : A \Rightarrow A$ and $(H, A) \rightarrowtail (\bar{e}, T, A')$
	
	\begin{itemize}
		\item goal: $\Gamma \vdash (e : A) \triangleright \bar{e} \Leftarrow A'$;
		\item given: $\Gamma \vdash A \Rightarrow e \Rightarrow B$ and $\Gamma \vdash A <: H$;
		\item by ind hypo, we can have $\Gamma \vdash e \Leftarrow B$;
		\item by \textcolor{oc-red-8}{Check Subsumption}, we can conclude $\Gamma \vdash e \Leftarrow A$, which is $\Gamma \vdash e : A \Rightarrow A$;
		\item use the \textcolor{oc-red-8}{Function Type Elimination} lemma, we can prove this case.
	\end{itemize}
\end{frame}

\begin{frame}{Proof of Soundness (Cont.)}
	\textbf{case lam 1:} $\Gamma \vdash \boxed{e_1} \mapsto H \Rightarrow \lambda x .~e \Rightarrow A \rightarrow B$ and $(\boxed{e_1} \mapsto H, A \rightarrow B) \rightarrowtail (\bar{e}, T, A')$
	\begin{itemize}
		\item by inversion, we have $(H, B) \rightarrowtail (\bar{e}, T, A')$ and $\Gamma \vdash Top \Rightarrow e_1 \Rightarrow A$;
		\item goal: $(\Gamma \vdash (\lambda x~.e) ~ e_1) \triangleright \bar{e} \Leftarrow A'$;
		\item by ind typo, we have $\Gamma , x : A \vdash e \triangleright \bar{e} \Leftarrow A'$ and $\Gamma \vdash e_1 \Leftarrow A$;
	\end{itemize}
\end{frame}

\begin{frame}{Proof of Soundness (Cont.)}
	\textbf{case lam 2:}	$\Gamma \vdash A \rightarrow B \Rightarrow e \Rightarrow A \rightarrow C$
	\begin{itemize}
		\item goal: $\Gamma \vdash e \Leftarrow A \rightarrow C$;
		\item proved by apply ind hypo.
	\end{itemize}
\end{frame}

\section{Completeness}

\begin{frame}{Completeness}

	I think the main issues are around the rule T-App2: For example, $(\lambda x . ~x)~1$.

	\begin{mathpar}
		\inferrule*[lab=T-App2]
		{\Gamma \vdash e_1 \Leftarrow A \rightarrow B \\
			\Gamma \vdash e_2 \Rightarrow A}
		{\Gamma \vdash e_1~e_2 \Rightarrow B}

		\inferrule*[lab=T-App2]
		{\Gamma \vdash \lambda x. ~x \Leftarrow Int \rightarrow Top \\
			\Gamma \vdash 1 \Rightarrow Int}
		{\Gamma \vdash (\lambda x . ~x)~1 \Rightarrow Top}
	\end{mathpar}

	The $B$ in rule T-App2 is too relaxed; we may include a \emph{minimal typing}.

	\begin{lemma}[Completeness (Inf)]
		If $\Gamma \vdash e \Rightarrow A$, then $\exists B, \Gamma \vdash Top \Rightarrow e \Rightarrow B$ and $\Gamma \vdash B <: A$.
	\end{lemma}

	\begin{lemma}[Completeness (Chk)]
		If $\Gamma \vdash e \Leftarrow A$, then $\exists B, \Gamma \vdash A \Rightarrow e \Rightarrow B$ and $\Gamma \vdash B <: A$.
	\end{lemma}

\end{frame}
%end region

\begin{frame}{Completeness (Gen)}
	\begin{theorem}[Completeness]
		If $\Gamma \vdash e \Leftrightarrow A$, then $\exists B, \Gamma \vdash (f \Leftrightarrow A) \Rightarrow e \Rightarrow B$ and $\Gamma \vdash B <: A$.
		\begin{align*}
			f \Leftarrow  A =  & ~A   \\
			f \Rightarrow A  = & ~Top
		\end{align*}
		\begin{proof}
			Induction on $\Gamma \vdash e \Leftrightarrow A$;
		\end{proof}
	\end{theorem}
	Note: we can further simplify the theorem (check condition) If we have the property that
	\begin{lemma}[Typing Implies Subtyping]
		If $\Gamma \vdash A \Rightarrow e \Rightarrow B$, then $\Gamma \vdash B <: A$.
	\end{lemma}
\end{frame}

\begin{frame}{Main issues}

	\begin{mathpar}
		\inferrule*[Right=T-App2]
		{\emptyset \vdash \lambda x. ~(\lambda y. ~y) \Leftarrow Int \rightarrow (Int \rightarrow Int) \\ \emptyset \vdash 1 \Rightarrow Int}
		{\emptyset \vdash (\lambda x. ~(\lambda y. ~y))~1 \Rightarrow Int \rightarrow Int}
	\end{mathpar}

	\begin{mathpar}
		\inferrule*[Right=T-App]
		{
			\inferrule*[Right=T-Lam1]
			{\emptyset \vdash Top \Rightarrow 1 \Rightarrow Int \\
				x : Int ~\cancel{\vdash} Top \Rightarrow (\lambda y. ~y) \Rightarrow Int \rightarrow Int}
			{\emptyset ~\cancel{\vdash} \boxed{1} \mapsto Top \Rightarrow \lambda x. ~(\lambda y. ~y) \Rightarrow Int \rightarrow Int}
		}
		{\emptyset ~\cancel{\vdash} Top \Rightarrow (\lambda x. ~(\lambda y. ~y))~1 \Rightarrow Int \rightarrow Int}
	\end{mathpar}

\end{frame}

\section{Cases}

\begin{frame}{$(\lambda x. ~x)~1$}
\begin{mathpar}
\inferrule*[Right=T-App]
{
\inferrule*[Right=T-Lam1]
{. \vdash Top \Rightarrow 1 \Rightarrow Int \\ x : Int \vdash Top \Rightarrow x \Rightarrow Int}
{. \vdash \boxed{1} \rightarrow Top \Rightarrow \lambda x. ~x \Rightarrow Int \rightarrow Int}
}
{. \vdash Top \Rightarrow (\lambda x. ~x)~1 \Rightarrow Int}
\end{mathpar}
\end{frame}

\begin{frame}{$(\lambda f . ~ f ~ 1) : ((Int \rightarrow Int) \rightarrow Int)~ (\lambda x. ~x)$}

\begin{mathpar}
\inferrule*[Right=T-App]
%%%%%% T-App Premise
{
\inferrule*[Right=T-Ann]
{
\inferrule*[Right=T-Lam2]{ }{. \vdash (Int \rightarrow Int) \rightarrow Int \Rightarrow \lambda f . ~ f ~ 1 \Rightarrow (Int \rightarrow Int) \rightarrow Int} \quad \quad \quad \quad
\inferrule*[Right=S-Arr]{ }{. \vdash (Int \rightarrow Int) \rightarrow Int <: \boxed{\lambda x. ~x} \rightarrow Top}
}
{. \vdash \boxed{\lambda x. ~x} \rightarrow Top \Rightarrow (\lambda f . ~ f ~ 1) : ((Int \rightarrow Int) \rightarrow Int) \Rightarrow (Int \rightarrow Int) \rightarrow Int}
}
%%%%%% T-App Conclusion
{. \vdash Top \Rightarrow (\lambda f . ~ f ~ 1) : ((Int \rightarrow Int) \rightarrow Int)~ (\lambda x. ~x) \Rightarrow Int}
\end{mathpar}

For the second premise, we have

\begin{mathpar}
\inferrule*[Right=S-Arr]
{
\inferrule*[Right=S-Tele]
{. \vdash Int \rightarrow Int \Rightarrow \lambda x. ~x \Rightarrow Int \rightarrow Int}
{ . \vdash \boxed{\lambda x. ~x} <: Int \rightarrow Int} \\
. \vdash Int <: Top
}
{. \vdash (Int \rightarrow Int) \rightarrow Int <: \boxed{\lambda x. ~x} \rightarrow Top}
\end{mathpar}
	
\end{frame}

\end{document}
